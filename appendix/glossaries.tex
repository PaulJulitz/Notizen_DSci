
%%%%%%%%%% Glossar %%%%%%%%%%%%%%%%

\newglossaryentry{g_Epoch}{
	name={Epoch},
	description={
	One epoch is when an ENTIRE dataset is passed forward and backward through the neural network only ONCE.
	}
}

\newglossaryentry{g_ONNX}{
	name={ONNX},
	description={
	Open Neural Network Exchange is an open ecosystem for AI models. It helps move you model with the different development cycle.
	}
}

\newglossaryentry{g_Measure}{
	name={Measure},
	description={
	Dies ist eine berechnenden DAX-Funktion. Dies ist zu unterscheiden zu den \textit{Berechneten Feld} Funktionen, die für Tabellen die nicht im Datenmodell enthalten sind, zu nutzen sind. 
	}
}

\newglossaryentry{g_Berechnende-Spalte}{
	name={Berechnende Spalte},
	description={
		Diese ergänzen Tabellen, welche im Datenmodell eingefügt sind. Dabei werden Daten erzeugt und auch im Arbeitsspeicher geladen. Diese Spalten lassen keine Aggregations zu und beziehen sich ausschließlich auf jede Zeile in einer Tabelle. 
	}
}

\newglossaryentry{g_IS}{
	name={Intellisense},
	description={
		Dies ist eine Anwendung zur Auto-Vervollständigung
	}
}

\newglossaryentry{g_API}{
	name={Application Programming Interface},
	description={
		Ein Programmschnittstelle ist eine Interaktionsmöglichkeit, die ein Programm anderen Programmen bietet auf bestimmte Teile der Software und oder Daten zuzugreifen.
	}
}

\newglossaryentry{g_ODBC}{
	name={Open Database Connectivity (or Connector)},
	description={
		Es handelt sich hierbei um eine API Verbindung zu einem DBMS.
		Das Ziel ist, dass eine Datenverbindung zu Datenbanksystem hergestellt werden kann, die unterschiedlicher Natur und oder auf unterschiedlichen Plattformen laufen. 
	}
}

\newglossaryentry{g_QlikSense}{
	name={Qlik Sense},
	description={
		Qlik Sense ist eine Business Analytics Software, welche die QlX-Engine Technologie nutzt.
	}
}

\newglossaryentry{g_M}{
	name={M Formula Language},
	description={
		Power Query nutzt die Sprache M. Diese ist eine funktional Sprache ähnlich zu F. Es können mit den Ausdrücken Daten gefiltert und transformiert werden.
	}
}

\newglossaryentry{g_SQLite}{
	name={SQLite},
	description={
		SQLite ist ein relationales Datenbanksystem. 
	}
}

\newglossaryentry{g_SQL}{
	name={SQL},
	description={
		\gls{SQL} ist eine Sprache welche strukturierte Abfragen an Datenbanken stellt. Diese Sprache wird von Datenbanken wie MySQL, Oracle, Mircosoft SQL Server, etc. verwendet. 
	}
}

\newglossaryentry{g_Git}{
	name={Git},
	description={
		Git is a free and open-source distribution \gls{VCS}. It trackes changes in source code during software development.  
	}
}

\newglossaryentry{g_EC2}{
	name={Amazon Elastic Computing Cloud},
	description={
		Amazon Elastic Compute Cloud (Amazon EC2) bietet eine skalierbare Rechenkapazität in der Amazon-Web-Services(AWS)-Cloud. Amazon EC2 beseitigt die Notwendigkeit, im Voraus in Hardware investieren zu müssen. Daher können Sie Anwendungen schneller entwickeln und bereitstellen. Sie können Amazon EC2 verwenden, um so viele oder so wenige virtuelle Server zu starten, wie Sie benötigen, Sicherheit und Netzwerk zu konfigurieren und den Speicher zu verwalten. Amazon EC2 können Sie auf- oder abwärts skalieren, um auf geänderte Anforderungen oder Datenverkehrsspitzen zu reagieren. Dies reduziert die Notwendigkeit, den Datenverkehr vorauszusagen.
	}
}

\newglossaryentry{g_Git_Repository}{
	name={Git Repository},
	description={
		Repositories in GIT contain a collection of files of various different versions of a Project. These files are imported from the repository into the local server of the user for further updations and modifications in the content of the file. A \gls{VCS} or the Version Control System is used to create these versions and store them in a specific place termed as a repository \cite{Geeks.2020}.  
	}
}

\newglossaryentry{g_Git_CMD}{
	name={Git Repository},
	description={
		Repositories in GIT contain a collection of files of various different versions of a Project. These files are imported from the repository into the local server of the user for further updations and modifications in the content of the file. A \gls{VCS} or the Version Control System is used to create these versions and store them in a specific place termed as a repository \cite{Geeks.2020}.  
	}
}

\newglossaryentry{g_Shell}{
	name={Shell},
	description={
		Eine Shell ist ein Befehl Interprator (Commandline Interpretar). Diese gibt Zugang zum Betriebssystem. Im Regelfall wird dieser über ein command-line or graphical user interface gesteuert.
	}
}

\newglossaryentry{g_Python}{
	name={Python},
	description={
		Python is an interpreted, high-level and general-purpose programming language. Python's design philosophy emphasizes code readability with its notable use of significant whitespace. \cite{wiki.pyh}
	}
}

\newglossaryentry{g_HTTP}{
	name={HTTP},
	description={
		The Hypertext Transfer Protocol is an application layer protocol for distributed, collaborative, hypermedia information systems. HTTP is the foundation of data communication for the World Wide Web, where hypertext documents include hyperlinks to other resources that the user can easily access, for example by a mouse click or by tapping the screen in a web browser. \cite{wiki.HTTP}
	}
}

\newglossaryentry{g_URL}{
	name={URL},
	description={
	A Uniform Resource Locator (URL), colloquially termed a web address, is a reference to a web resource that specifies its location on a computer network and a mechanism for retrieving it. A URL is a specific type of Uniform Resource Identifier (URI), although many people use the two terms interchangeably. URLs occur most commonly to reference web pages (http), but are also used for file transfer (ftp), email (mailto), database access (JDBC), and many other applications. \cite{wiki.URL}
	}
}

\newglossaryentry{g_JSON}{
	name={JSON},
	description={
		JavaScript Object Notation strukturiert Daten. Dieses Datenformat hat sich in der Webentwicklung verbreitet. Bei Anfragen von Rest-API kann man ein JSON Format erwarten. Ebenso finden JSON-Formate in Log und Confi-Datein ihren Nutzen.
		Es gibt 6 Datenformate, welche in einer JSON Datei abgespeichert werden können. String, Boolom, Integer, null, Array und Object
	} 
}

\newglossaryentry{g_NULL}{
	name={NULL (R)},
	description={
		In \gls{CPP} oder R wird \textit{NULL} verwendet, wenn Variablen deklariert werden. Der Pointer zeigt auf \textit{NULL}, und die Variablen bleiben vorherst leer: \textit{int height;}.
	} 
}

\newglossaryentry{g_None}{
	name={None},
	description={
		None ist ählich zu NULL. Es ist ein eigener Datentype: NoneType. Dieser Datentyp kann nur einen Wert besitzten \textit{None}. Zum Vergleich Integer kann Werte von $-3, 2, 10, \dots$ besitzen. Aus diesem Grund ist $None == None$ \textit{true}. Ebenso ist $None$ \textit{is} $None$ \textit{true}, weil es nur ein Objekt gibt, welches diesen Wert besitzt.
		In Python werden Varialben \underline{nicht} zuerst deklariert. Diese bedeuetet, diese halten immer eine Wert in dem entsprechenden Typ.
		Funktionen, welche keinen Rückgabewert wieder geben, nutzen \textit{None}: def no-return(void): pass $\rightarrow$ print(no-return):\textit{None}. Die normale Ausgabe der Funktion zeigt nicht an. Diese hat damit zu tun, dass Python dies unterdrückt.\\
		Die Instanz \textit{None} gibt es in Python nur einmal, wird mehrere Objekte \textit{None} zugewiesen, so zeigen alle Objekte, auf die selbe Instanz. \footnote{\href{https://medium.com/analytics-vidhya/dealing-with-missing-values-nan-and-none-in-python-6fc9b8fb4f31}{Link}}
	} 
}

\newglossaryentry{g_NaN}{
	name={NaN},
	description={
		NaN ist eine numerische Fehlerabfrage. Es ist ein besonder \textit{floot}-Zeiger. Dieser kann aus verschiedensten Gründen auftretten. Der Unterschied zu None ist, dass None bewusst definiert und ausgegeben wird - ähnlich zu Null. Bei NaN handelt sich sich um ein Float wert, der aus den verschiedensten Gründen entstehen kann, immer aber eine Sackgasse representiert, weil von NaN zu anderen Werten nicht transformiert werden kann - Virusbefall von Daten.\\
		Aus diesem Grund ist ein Vergleich von zwei NaN Ausgaben auch nicht zwangläufig gleich, weil die interne Kodierung von Fall zu Fall unterscheidlich ist. Es kann ebenso sein, dass die Sprache NaN direkt definiert - wie in Python, dass eine Gleichungs-Abfrage $NaN==NaN \rightarrow$ \textit{false} zurückgibt. Dies würde dann bedeuten, selbst wenn der Fehlercode der gleiche ist, würde \textit{false} zurückgegeben. \\
		Was die Identität $id()$ angeht, sind Variablen mit $np.nan$ gleiche Objekte. Zwischen Dataframe erstellt NaN, NumPy, Math, Etc. können und sind meistens die Objekt-Identitäten anderen, weshalb eine Abfragen von Dataframe erstellten NaN und NumPy NaN nicht in $true$ sondern $false$ resultiert. %\footnote{\href{Webseite}{https://towardsdatascience.com/navigating-the-hell-of-nans-in-python-71b12558895b}}
	} 
}

\newglossaryentry{g_Na}{
	name={Na (R)},
	description={
		In R exitiert zu NULL und NaN auch Na. Es handelt sich hier um einen logischen Operator. Die Besonderheit ist, dass dieser, wenn abgefragt, $Na$ zurückgibt. $Na=Na \rightarrow Na$. Dieser Operator kann in verschieden Datetypen: Array, Vektor, Matrix, etc. vorkommen.
		\footnote{\href{https://www.r-bloggers.com/2018/07/r-null-values-null-na-nan-inf/}{Link 2}\href{https://www.r-bloggers.com/2010/04/r-na-vs-null/}{Link}}
	} 
}

\newglossaryentry{g_scikit}{
	name={scikit learn},
	description={
		Es enthält sich um ein Packet, welches Algorithmen für maschninielles Lernen in Python beinhalten. Fragestellungen aus dem Data Scince Bereich können damit beantwortet werden. Ebenso enthält scikit learn zusätzliche Funktionen die bei Preprocessing, Feature Extraction, Evaluation und weiteres helfen.
		Dier Name setzt sich aus Science und Toolkit zusammen. Abgekürzt heißt es sklearn.
	} 
}

%%%%%% Stochastik 


\newglossaryentry{g_Stochastik}{
	name={Stochastik},
	description={
		Unter dem Feld der Stochastik wird die \textbf{Wahrscheinlichkeitstheorie} und die \textbf{Statistik} zusammengefasst. Der Begriff Stochastik bedeutet \textit{Kunst des Vermutens}. Zwei Schwerpunkt des Wahrscheinlichkeitslehre ist, zu modellieren, wie wahrscheinlich es ist das ein Ereignis eintritt und welche Ereignise es geben kann. Statistik hat den Schwerpunkt anhand von Daten Rückschlüsse auf Charakteristika eine wahrscheinlichkeitstheoritischen Modells zu ziehen.
	} 
}

\newglossaryentry{g_Statistik}{
	name={Statistik},
	description={
		Das Feld der Statistik ist ein Teilbereich der Stochastik. Der Schwerpunkt der Statistik ist, Datenmenge auszuwerten, zu interpretieren und wie diese verteilt sind.
	} 
}

\newglossaryentry{g_Wahrscheinlichkeitstheorie}{
	name={Wahrscheinlichkeitstheorie},
	description={
		Die Wahrscheinlichkeitstheorie, auch Wahrscheinlichkeitsrechnung oder Probabilistik, ist ein Teilgebiet der Mathematik, das aus der Formalisierung, der Modellierung und der Untersuchung von Zufallsgeschehen hervorgegangen ist.
	}
}

% Up
%%%%%% Stochastik 

\newglossaryentry{g_BuildArtifacts}{
	name={Build Artifacts},
	description={
		In Kurz: Environment + Compieled Output = Artifact. 
		In Lang: Die Umgebung mit alle Werkzeugen, Abhängikeiten, welche benötigt werden, um die Build (oder Image oder Source) plus tatsächlich compilierte resultate. Kommt es zu einem Fehler, ist alles an einem Ort, um unabhängig von den Konfigurationen, eine Fehleranalyse durchzuführen.
		Artifact kann auch nur $"$etwas produziertes$"$ bedeuteten. Somit ist \textit{libs/runnables - Compilied Artifacts}; \textit{Image - Build Step}.
	}
}

\newglossaryentry{g_conda}{
	name={conda},
	description={
		Pip installs Python packages whereas conda installs packages which may contain software written in any language. For example, before using pip, a Python interpreter must be installed via a system package manager or by downloading and running an installer. on the other hand can install Python packages as well as the Python interpreter directly.
	}
}

\newglossaryentry{g_pip}{
	name={pip},
	description={
		Package Installer for Python is the de facto and recommended package-management system written in Python and is used to install and manage software packages. It connects to an online repository of public packages, called the Python Package Index.
	}
}


\newglossaryentry{g_DataLake}{
	name={Data Lake},
	description={
		Es besteht die Chance, das Data Lake und Data Warehouse im gleichen Kontext verwendet werden. Diese beschreiben jedoch zwei unterschiedliche Konzepte.
		Ein Daten in einem Data Lake haben kein spezifischen Anwendungsfall, in welchen erkannt wird, in welchem diese genau benötigt werden. Ein Data Warehouse im Gegenzug hat einen initalen Anwendungsfall, welche eine genaue Verbindung zwischen Daten und der Anwendung erkennen lässt. In einem Data Lake werden die Daten unstrukturiert, ungefiltert und ohne einen Qualitätscheck abgelegt. Dies Daten werden von Data Scientist und Data Analyist herangezogen, um mit ML-Algorithmen neue Erkenntnisse zu gewinnen. Es handelt sich hierbei um eine neuere Technologie.
	}
}

\newglossaryentry{g_DataWarehouse}{
	name={Data Warehouse},
	description={
		Es besteht die Chance, das ein Data Warehouse und ein Data Lake im gleichen Kontext verwendet werden. Im Gegensatz zu einem Data Lake können die gespeicherten Daten einen eindeutigen Anwendungsfall zugeordnet werden. Die Daten werden bereinigt und stehen in verwendbarer Qualität zur Verfügung. Diese Daten werden meist von Business Analysten verwendet. Diese Technologie ist eine etablierte Technologie.
	}
}

\newglossaryentry{g_Objective(OKR)}{
	name={Objective (OKR)},
	description={
		Der Begriff $"$Objective$"$ ist für mich in Anlehnung an die \gls{OKR} Logik. Anders als im englischsprachigen Militär ist dies ein Zustand, welcher erstrebenswert ist, welcher keine eindeutige Zielerreichung beinhaltet muss. Für diesen Zustand besteht die Möglichkeit, diesen Zustand kontinuierlich anzupassen und abzugleichen. Zum Beispiel: \textit{Mein Ziel ist, dass Deutschland ein demokratisches Land ist.} Was Demokratie ausmacht, kann zum Anfang klarer definiert sein, im Laufe der Zeit jedoch sich ändern und mit anderen abgeglichen werden. Bei einem \textit{Objective} ist es daher umso wichtiger, dass \textit{Key-Resultate} und \textit{Maßnahmen} genau beschrieben werden. Sie geben an, welche Indikatoren man sich selbst setzt oder annimmt, was den Erfolg der Maßnahmen ausmacht oder wie das \textit{Objective} definiert ist. Die Maßnahmen beschreiben das genaue Vorgehen, wie was getan wird, um das \textit{Objetive} schlussendlich zu erreichen.
	}
}

\newglossaryentry{g_Kernel_Jy_Py}{
	name={Kernel (IPython/ Jupyter)},
	description={
		The IPython kernel (ipykernel) is a Jupyter Notebook (web-interface) kernel for Python execution. This kernel is used to execute Python notebook code.
	}
}

\newglossaryentry{g_Kernel_Jy}{
	name={Kernel (General Jupyter)},
	description={
		The kernel is in general a the engine to excute the markdown and compute commands in a Jupyter Notebook (web-interface). Aka Compution environment
	}
}

\newglossaryentry{g_PyI}{
	name={Python Interpreter},
	description={
		The Python interpreter is CPython and is written in C programming languages. The interpreter checks and anaylses the source code. This code gets transformed into byte and then bit code. After evaluating it, the output gets returned.
	}
}

\newglossaryentry{g_JupyterNotebook}{
	name={Jupyter Notebook (Web Interface)},
	description={
		A Jupyter Notebook is a web-interface for edition code and text. Colloaquialy the file type is called Jupyter Notebook too. Also Jupyter Notebook gets reference for the set of objects to run the web interace and the underlining substructruke like a kernel.
	}
}


%%%%%%%%%%%%%%%%%%%%%%%%%%%%%%%%%%%%%
%\glsaddall % Der Befehl fügt alle Glossary-Einträge in das Glossar ein.