\pagebreak
\chapter{Erläuterung}

\section{Strategie} \label{Appendix_Erlaeuterung_Strategie}

\subsection{Erläuterung}
\paragraph{Definition}
Eine Strategie besteht aus einer Menge aus \textit{integriert} Entscheidungen, welche einem auf einem Spielfeld bestmöglich für einen Erfolg positioniert.\\

% Stategy - A integrated set of choices, which position you on a playing field of you choice to win.

%% Theory: This theory has to be coherent and doeable.:
%%% Why are we on this playing field on not on a other one.
%%% How we are on this playing field better the anyone else on serving the customer.

%% Difference to planing: To coherent structure is nessesary. In planing, you choice activity where the outcome is determinde by you. The strategy makes prediction about what we are going to to to archive a certain outcome.


\paragraph{Eine Menge von integrierten Entscheidungen} sind Entscheidungen, welche von durch ein kohärente Theorie verwendet werden, welches erklärt warum und wie diese einen bestmöglich positionieren werden. Dabei kann sich bestehender Wirkungszusammenhänge bedient werden oder angenommene herangezogen werden, welche überprüft im weiteren Verlauf überprüft werden. Um das Spielfeld oder alternativ die Umgebung, den Markt einzuschätzen, ist eine Analyse notwendig. Dabei sollen die Regeln, Chancen und Risiken identifiziert werden, welche für das Agieren relevant sind. Ebenso ist es relevant zu erkennen, welche Wettbewerber gleiche Zielsetzungen verfolgen und ob diese gleiche Restriktion sich ausgesetzt sehen.\\

\paragraph{Unterschied zum Plan:} Ein Plan erfordert keine koheränte Struktur, welche die Entscheidungen an einem Ziel ausrichtet oder erklärt, warum genau diese Entscheidung schlussendlich zum Erfolg führt. Es kann sein, dass die Entscheidungen genau die gleichen sind, welche durch eine überlegte Strategie auch hätten ausgewählt wurden. Eine Strategie ist jedoch selten statisch und es erfordert Anpassungen der Strategie, wenn neue Information sich auftuen. Zum Beispiel die Annahmen über das Spielfeld sind nicht vollständig oder richtig. Jetzt zeigt sich der Vorteil einer Strategie als im Vergleich zu einem Plan. Eine Strategie legt nahe, warum genau die Entscheidungen Anhand der Spielfeld Gegebenheiten getroffen wurden. Ändern sich die Gegenbenheiten, so kann die Wirkung der Entscheidungen neu analysiert werden.

\subsection{Arten der Entscheidung}
 Entscheidung über Maßnahmen, Prinzipien oder Objectives}

\paragraph{Maßnahme} wird hier als ein 
\begin{itemize}
	\item \textbf{konkreter},
	\item \textbf{expliziter}
	\item und \textbf{abgeschlossener} Schritt
\end{itemize}
definiert.\\

Im oberen Bezug auf die Strategie, wird sich bezüglich einer Maßnahme entschieden: Eine Maßnahme kann ein Projekt, eine Software, eine eigene Strategie sein. Wirkungskette: Wichtig dabei, ist zu erklären, offen zu legen, wie diese Maßnahme auf das obergeordnete Ziel einzahlt.
Je nach Detailgrad kann hierbei
\begin{itemize}
	\item aufgeführt werden, welche exekutiven Schritte gegangen werden, um die abgeschlossenen Maßnahme zu erreichen,
	\item und ebenso eigene Ziel festlegen, welche die Maßnahme abgeschlossen bewertbar machen.
\end{itemize}
In der Stategieplanung kann diese substruktur wiederum auf die großen Ziele bewertet werden.

\paragraph{Prinzip: Hilfestellung für zukünftige Entscheidungen}
Können noch keine Entscheidungen bezüglich Maßnahmen getroffen werden, kann das Werkzeug \textbf{Prinzip} zum Einsatz kommen.\\

Hier gilt die Definition einer Strategie ebenso. Wird sich für bestimmte Prinzipien entschieden, müssen diese ebenso in einen kohärenten Rahmen eingebaut werden, warum diese Prinzipien dabei helfen, auf die Ziele einzuzahlen - Wirkungskette.\\

Wie erläutert, geht es bei einer Strategie integrierte $"$Handlungsentscheidungen$"$ treffen zu können, welche einem näher zu seinem Ziel bringen. Dies können schon ausgefertigte Maßnahmen sein \underline{oder} Prinzipien, welche in bisher unbekannten aber erwartbaren Situationen eine Richtung vorgeben, wie sich in der Zukunft entschieden werden soll, um die Ziele zu erreichen, oder sich für Maßnahmen in der Zukunft entscheiden zu können.

\paragraph{Objectives reichen machmal aus, ist aber keine Strategie}
Das Ziel einer Strategie, ist es die gesetzten Ziele zu erreichen, in dem integrierte Entscheidungen vorliegen, welchen einen so positionieren, die Ziel bestmöglich zu erreichen. Es kann jedoch auch so sein, dass die Ziele selbst als Ausrichtung für zukünftige Entscheidungen (z.B.: Abwägungen) treffen zu können.\\

Damit wir von einer Strategie sprechen können, sollte wenigsten Zielkonflikte und -abhängigkeiten analysiert werden. Diese könnte aber auch die gleiche Wirkung haben - dass sie ausreichen die jeweiligen Ziele zu erreichen.\\

Ein persönliches Beispiel: Für die eigenen Ausrichtung im Leben, habe ich drei Ziele definiert. Zu diesen Zielen sind weder Maßnhahmen oder tiefergelegte Prinzipien festgelegt.

\section{Erläuterung Prinzip und Mechanismus}\label{Appendix_Erlaeuterung_Prinzip} Prinzipien und Mechanismen sind eng miteinander verbunden, da Prinzipien oft die zugrunde liegenden Konzepte oder Ideen sind, die die Funktionsweise eines Mechanismus bestimmen.\\

Ein Mechanismus ist die konkrete Umsetzung oder die praktische Realisierung eines Prinzips. Das Prinzip legt fest, wie etwas funktionieren soll, während der Mechanismus die spezifische Art und Weise beschreibt, wie dies tatsächlich erreicht wird.\\

Zum Beispiel könnte das Prinzip der Schwerkraft besagen, dass alle Massen auf der Erde von der Gravitationskraft angezogen werden. Der Mechanismus, der dies umsetzt, könnte die Bewegung von Objekten durch die Krümmung der Raumzeit gemäß der Allgemeinen Relativitätstheorie sein.\\

In diesem Zusammenhang liefert das Prinzip den theoretischen Rahmen, während der Mechanismus die praktische Umsetzung oder Realisierung dieses Prinzips beschreibt.




