\section{Azure ML Python Development}\label{sec:PythonDevEnv}
\subsection{Understanding ipykernel}
\subsubsection{Overview Python Interpreter}
Reference: \href{https://docs.python.org/3/tutorial/interpreter.html#the-interpreter-and-its-environment}{Using the Python Interpreter}, \href{https://blog.hubspot.com/website/what-is-python-interpreter#:~:text=A%20python%20interpreter%20is%20a,and%20low%2Dlevel%20languages%20are.}{What Is a Python Interpreter?}\\


\paragraph{High-Level Language}
The \gls{g_Python} language is a high-level programming language. This means, that the programming language is more similar to human language then to machine language, which is written in strings of bits - ones and zeros.
The advantage of it is that it is better understood by humans, but the same can not be said for the machines. Writing commands (instruction) in lines of zeros and ones is more difficult, that using higher level concepts. The gap between this, gets closes by the \textbf{Python Interpreter}.\\

The \gls{g_PyI} reads the command written by the \gls{g_Python} programmer, evaluates them, and returns the output. If the \textit{python.exe} application/ interpreter is opened, either through the \gls{CMD}, for example  
\begin{lstlisting}[style=CMD]
	$ python3.12
	///Python 3.12 (default, April 4 2022, 09:25:04)
	///[GCC 10.2.0] on linux
	///Type "help", "copyright", "credits" or "license" for more information.
	>>>
\end{lstlisting}
the commands
\begin{lstlisting}[style=CMD]
	>>> the_world_is_falt = True
	>>> if the_world_is_flat:
			print("Be careful not to fall off!")
	/// Be careful not to fall off!
\end{lstlisting}
or by opening the application directly.
\begin{figure}[H]
	\centering
	\includegraphics[scale = 0.2]{attachment/chapter_AML/Scc003}
	\caption{Using the command line with the python application/ interpreter}
\end{figure}
In both cases, the \gls{g_PyI} receives the commands and executes them. In the case where the \gls{IDE} is to write a the Python script (aka. module or application), the scripts gets passed to \gls{g_PyI}.\\

\paragraph{Difference between interpreter and a compiler}
Both the interpreter and compiler transforms the source code into binary machine code.
The difference arise through the way they are doing it differently: The interpreter translate the source code one statement at a time. The compiler on the other hand first scans the entire programm and then translate the whole program into machine code. For more detail, see \href{https://blog.hubspot.com/website/what-is-python-interpreter#:~:text=A%20python%20interpreter%20is%20a,and%20low%2Dlevel%20languages%20are.}{What Is a Python Interpreter?} or \href{https://www.analyticsvidhya.com/blog/2021/05/choose-best-python-compilers-for-your-machine-learning-project-detailed-overview/#:~:text=What%20is%20a%20Python%20compiler,executed%20directly%20by%20a%20computer.}{Best Python Compiler}
\begin{figure}[H]
	\centering
	\includegraphics[scale = 0.3]{attachment/chapter_AML/Scc004}
	\caption{Example Compiler and Interpreter in Java}
\end{figure}

\subsubsection{ipykernel (Jupyter Notebook)}
Reference: \href{https://www.reddit.com/r/learnprogramming/comments/imhxai/what_is_the_difference_between_a_python_kernel_as/#:~:text=Python%20kernel%20is%20just%20a,is%20also%20not%20an%20interpreter.}{Difference Kernel and Interpreter}, \href{https://plotly.com/python/ipython-vs-python/}{IPython vs Python in Python}, \href{https://www.reddit.com/r/learnprogramming/comments/imhxai/what_is_the_difference_between_a_python_kernel_as/#:~:text=Python%20kernel%20is%20just%20a,is%20also%20not%20an%20interpreter.}{reddit:  difference between a Python Kernel (as in Jupyter notebooks) and a Python interpreter (like in PyCharm)?}, \href{https://python-forum.io/thread-40721.html}{jupyter kernel is an interface to the Python interpreter.} 
\\

\paragraph{IPython (Python3)}
To understand the \gls{g_Kernel_Jy_Py} it is helpful to understand \textit{IPython (Notebook)}. The \gls{g_PyI} passes command to it. This only allows commands for \gls{g_Python}.\\


\textit{IPyhton} creates an interactive command line terminal for \gls{g_Python}.  

\begin{figure}[H]
	\centering
	\includegraphics[scale = 0.3]{attachment/chapter_AML/Scc005}
	\includegraphics[scale = 0.2]{attachment/chapter_AML/Scc009}
	\caption{IPython interactive command-line terminal}
\end{figure}

\paragraph{Jupyter Notebook}
With the reorganization of \textit{IPython}, the new tool \textbf{Notebook} has been created. Under the project name \textbf{Jupyter}. This \gls{g_JupyterNotebook} is a web interface for \gls{g_Python}. It has the same interactive interface kept. Being a web-interface, it can integrate with many of the existing web libraries for data visualization.\\

The concept of a \textit{kernel} comes into play as the engine behind the web interface. The \textit{IPython} is now the backend with the \gls{g_Kernel_Jy_Py} for \gls{g_Python}. The \gls{g_Kernel_Jy_Py} with the advent of \gls{g_JupyterNotebook} is able to handel \textit{markdown} and \LaTeX text input.\\
\begin{figure}[H]
	\centering
	\includegraphics[scale = 0.3]{attachment/chapter_AML/Scc006}
	\caption{JupyterLab for an Enhanced Notebook, \href{https://realpython.com/using-jupyterlab/}{Jupyter Lab}}
\end{figure}
Note: The interaction with the \gls{g_Kernel_Jy_Py} is done through the \gls{g_JupyterNotebook}. With the latest tool: \textbf{Jupyter Lab}, interaction with separate files is possible.\\

$"$Inside$"$ the \gls{g_Kernel_Jy_Py} lives the \gls{g_PyI}. Another way of saying this is that the \gls{g_Kernel_Jy_Py} is the interface to the \gls{g_PyI}.\\

Note: \textit{Jupyter Lab} allows for a more interactive and simultaneous way to code.

\subsubsection{Interacting with different (language) Kernels}
\paragraph{Theoretical Multi languages Kernel}
As said before the \gls{g_Kernel_Jy_Py} is for interacting with the \gls{g_PyI} and the other functionality of a \gls{g_JupyterNotebook}.\\

The community around, \textit{Juypter Notebook}, the application, developed more \gls{g_Kernel_Jy} for the \gls{g_JupyterNotebook}. Those \gls{g_Kernel_Jy} allows to interact with different languages like \textit{Ruby, Scale, R}.\\

It would be possible (Unclear how) to create a \gls{g_Kernel_Jy} which allows to interact with all those languages in one \gls{g_JupyterNotebook}. The current research for this chapter could not found one. For example the \gls{g_Kernel_Jy_Py} allows us to interact for example with \gls{g_Python}, Markdown, Shell Script. This does not conclude for example \gls{SQL}.

\paragraph{VSCode Compute Cells Confusion}
If a \gls{g_JupyterNotebook} is open, for each cell it is possible to select the language for this cell. This those two things. First it changes the \textit{IntelSence} syntax highlights. Secondly, it provides the \gls{g_Kernel_Jy} information about the lanuage which is used in this cell.

\begin{figure}[H]
	\centering
	\includegraphics[scale = 0.3]{attachment/chapter_AML/Scc007}
	\caption{Cell Language Mode selection.}
\end{figure}

However, this \underline{does not mean} that all languages are supported by the selected \gls{g_Kernel_Jy}. 

\paragraph{Azure Data Studio}
In Azure Data Studio you can connect to different \gls{g_Kernel_Jy}, \href{
https://learn.microsoft.com/en-us/azure-data-studio/notebooks/notebooks-guidance?view=sql-server-ver15}{Azure Data Studio}:
\begin{itemize}
	\item \gls{SQL} Kernel,
	\item PySpark3 and PySpark Kernel,
	\item Spark Kernel,
	\item Python Kernel (for local development)
\end{itemize}


\subsubsection{VEN and ipykernel}
% Spun up by Jupyter 
% Spn up on your machine
\paragraph{Spinning Up Jupyter Notebook with Anaconda}
\begin{itemize}
	\item First a \gls{g_conda} \gls{VEN} gets created.
	\item The Jupyter Notebook $"$application$"$ gets installed
\end{itemize}
This in turn will install the package for the web interface, \textit{IPython} and the default \gls{g_Kernel_Jy_Py} for \gls{g_Python}.
\begin{lstlisting}[style=CMD, caption={pip commands to get ready for Jupyter Notebok},captionpos=b]
	pip install jupyter
\end{lstlisting}
This command will import all dependence to work with a \gls{g_Kernel_Jy_Py}. An example of package environment see below.

\begin{lstlisting}[style=CMD, caption={Example Jupyter Notebook conda ven},captionpos=b]
	PS C:\Users\PaulJulitz\iCloudDrive\TexMaker\GitHub_Notizen_DSci\Notizen_DSci> conda list
	# packages in environment at C:\Users\PaulJulitz\anaconda3\envs\VSCode:
	#
	# Name                    Version                   Build  Channel     
	...
	azure-common              1.1.28                   pypi_0    pypi      
	azure-core                1.27.1           py38haa95532_0  
	azure-identity            1.12.0                   pypi_0    pypi      
	azure-keyvault-secrets    4.6.0                    pypi_0    pypi      
	azureml                   0.2.7                    pypi_0    pypi      
	...
	ipykernel                 5.3.4            py38h5ca1d4c_0
	ipython                   7.27.0           py38hd4e2768_0
	ipython_genutils          0.2.0              pyhd3eb1b0_1
	...
	jupyter_client            7.0.1              pyhd3eb1b0_0
	jupyter_core              4.8.1            py38haa95532_0
	jupyter_server            1.4.1            py38haa95532_0
	jupyterlab                3.2.1              pyhd3eb1b0_1
	jupyterlab_pygments       0.1.2                      py_0
	jupyterlab_server         2.8.2              pyhd3eb1b0_0
	...
\end{lstlisting}


\paragraph{Multiple Kernel (Python)}
If the setup is done through the Anaconda interface (\gls{g_conda}). The complete installation is done by Anaconda if the \gls{IDE} \gls{g_JupyterNotebook} is installed. By default \textit{IPython} is installed with the \gls{g_Kernel_Jy_Py}.\\

The command for the installation of the \gls{g_Kernel_Jy_Py} is
\begin{lstlisting}[style=CMD, caption={pip to install ipykernel},captionpos=b]
	pip install ipykernel
\end{lstlisting}
For example to use \text{Scala}, \href{https://community.databricks.com/t5/data-engineering/how-can-we-run-scala-in-a-jupyter-notebook/td-p/17752}{Link}:
\begin{lstlisting}[style=CMD, caption={Using Scala for a Notebook},captionpos=b]
	# Install the package
	pip install spylon-kernel
	# To select the kernel in the notebook, create a kernel spec
	python -m spylon_kernel install
	# Start Jupyter Notebook
	ipython notebook
\end{lstlisting}
In the notebook the kernel can be selected. The command is to list the available kernels is
\begin{lstlisting}[style=CMD, caption={pip to install ipykernel},captionpos=b]
	$ jupyter kernelspec list
	>>>>Available kernels:
	>>>>python2.7        /Users/jakevdp/.ipython/kernels/python2.7
	>>>>python3.3        /Users/jakevdp/.ipython/kernels/python3.3
	>>>>python3.4        /Users/jakevdp/.ipython/kernels/python3.4
	>>>>python3.5        /Users/jakevdp/.ipython/kernels/python3.5
	>>>>python2          /Users/jakevdp/Library/Jupyter/kernels/python2
	>>>>python3          /Users/jakevdp/Library/Jupyter/kernels/python3
\end{lstlisting}
See, \href{https://stackoverflow.com/questions/39007571/running-jupyter-with-multiple-python-and-ipython-paths}{Running Multiple Kernel} for understand how to select a \gls{g_Kernel_Jy} throught the command line. \\
% English
% for or to?-----
% She studied FOR the exam. "For" indicates the goal, which should be achived.
% She went TO study. "to" indicates the use of a purpose before a infinit verb.

To see which \gls{g_Kernel_Jy} is running, run the following code
\begin{lstlisting}[style=python]
import sys
print(sys.executable)
print(sys.version)
print(sys.version_info)
\end{lstlisting} 

\subsubsection{Connecting a Notebook to a Compute Ressources}
%\paragraph{Locally}
To use a \gls{g_JupyterNotebook}, the web interface creates either a
\begin{itemize}
	\item local server,
	\item or provides the possibility to connect to a Jupyter hosted server.
\end{itemize}
Note: The local server is still your own machine.\\

The packages for connecting to a server are installed in the \gls{VEN}. The "spinning" up is basically the configuration to either a local port on your machine or to a hosted server. The former is used for computing.

\begin{figure}[H]
	\centering
	\includegraphics[scale = 0.3]{attachment/chapter_AML/Scc008}
	\caption{Mac Setup of Jupter Notebook Server}
\end{figure}

If the notebook should be connected to the compute cluster, this is done by a SHH Tunnel.

\begin{figure}[H]
	\centering
	\includegraphics[scale = 0.3]{attachment/chapter_AML/Scc010}
	\caption{Tunnel Trafic, \href{https://radcamp.github.io/AF-Biota/Jupyter_Notebook_Setup.html}{SHH Tunnel}}
\end{figure}


\subsection{Basics Azure ML SDK}
