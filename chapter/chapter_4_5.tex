\section{Epolog}

\subsection{Concept Mapping}

Im weiteren Umgang mit dem \gls{p_NFM} Vorhaben, wird es viele weitere Gedanken dazu geben. Dieser Abschnitt soll dabei helfen, zwischen neuen Gedanken zu unterscheiden. Sind diese als leicht anderes Konzept zu verstehen oder als Erweiterung der bisherigen Logik oder sogar Revision der bestehenden. Bevor somit ganze Bereiche umgeschrieben werden, soll darüber nachgedacht werden, ob der neue Gedanke nur ein leicht anderes Konzept representiert, welches nur sprachlich gemappt werden soll, oder wirklich Änderungen an der bestehenden Arbeit vorgenommen werden muss.


\subsubsection{Ungleichgewicht}

Es besteht die Annahme, dass wenn keine Gegenmaßnahmen ergriffen oder ein Bewusstsein für diese Ungleichgewicht geschaffen wird, es außer Kontrolle gerät oder ewig verhart, welches \nameref{NFM_O_3} gefährdet. 
Das Risiko besteht, wird diese Ungleichgewicht nicht rechtzeitig reduziert oder nicht weise eingesetzt wird, um \nameref{NFM_O_1} zu erreichen, dass \nameref{NFM_O_3} nicht erreicht wird.\\

\subsection{Miss-Fits}\label{subsec_MissFits}
%Themen, welche nicht sauber reinpassen, aber im Kontext der Strategie beantwortet wird.
\subsubsection{Bedingungslose Liebe in der EKB} Wie kann dies unter Berücksichtigung der drei \gls{p_Obj} verstanden werden?\\

% Die Chance, immer eine Rückfall-Ebene zu haben
% Das bedeutet aber nicht, dass man auch gemocht wird und Leute gern mit einem Zeitverbringen wollen (Freundschaft) oder die Leistungen von einem wertschätzen (Drive))

Dieser Bestandteil der \gls{p_EKB} kann sich positive auf \nameref{NFM_O_3} auswirken, muss es aber nicht.\\
Es ist ebenso möglich, dass sich keine positive Beziehung mit natürlicher Sympathie einstellt, in welche die \gls{p_FM} sich gegenseitig Kraft geben. Ist dies der Fall, soll für \gls{p_NFM} sich jedoch gewiss fühlen, dass wir immer der eine bedingungslose Liebe zwischen den \gls{p_FM} herrscht.\\

Dabei ist es für \nameref{NFM_O_1} wichtig zu verstehen, dass dies nicht bedeutet, dass trotz keiner Anstrengung, alle einen sympathisch finden, gern mit einen Zeit verbracht wollen oder Respekt gegenüber den Leistungen vorherrscht. Es bedeutet, dass immer Hilfe und Unterstützung von den \gls{p_FM} einen entgegen gebracht wird und immer eine besonderen Beziehung zwischen den Eltern und \gls{p_NFM} vorherrscht.\\

Es besteht das Risiko, dass bei einer permanenten Rückfall-Ebenen, sich kein eigener Antrieb entwickelt. Wie im vorherigen Absatz beschrieben, wird damit nicht der Respekt für die eigenen Leistungen abgedeckt oder sich automatische eine \gls{p_FB} einstellen, in welcher die \gls{p_FM} gern mit einem Zeit verbringen wollen. Ist \gls{p_NFM} dies wichtig, besteht für es keine Garantie, dass dies Teil in der \gls{p_EKB} oder in der \gls{p_FB} wird.

\subsubsection{\gls{p_FB}} Neben der \gls{p_EKB} gibt es jedoch auch die \gls{p_FB}, der \gls{p_FM} untereinander. Beide Beziehungen können zwischen den \gls{p_FM} zur gleichen Zeit existieren.

\subsubsection{Wettbewerb mit uns (NFM)}
Unsere Strategie ist kein Geheimnis. Ganz im Gegenteil, es ist kein Grund bekannt, welche uns in eine schlechtere Lage versetzt, diese Strategie zu teilen. Hierbei ist zu vermerken, dies ist nicht immer der Fall: Besonders im Wettbewerb mit anderen Parteien kann es von Nachteil sein, wenn andere Parteien die Vorhaben kennen, besonders, wenn all das gleiche Ziel haben.\\

In unserem Fall stehe wir nicht mit anderen im Wettbewerb, sondern mit uns im Zeitverlauf. Unser Ziel, siehe \ref{part:NFM_sec:Objective}, ist nicht zeitgebunden jedoch außerhalb unsere Kontrolle. Dennoch können wir $"$verlieren$"$. Dies würde zu uns selbst passieren. Wir erreichen nicht den gewünschten Zustand in unserem Leben, welcher uns die größte Freude und Erfüllung bereitet - und warum sollte man danach nicht streben. Es gibt jedoch genügend Irrwege, Hindernisse und Risiken, welche uns daran hindern, unseren Wunsch-Zustand zu erreichen. Diese sollen in einer strategischen Betrachtung erkannt und adressiert werden. Welches nicht bedeutet, dass diese immer gelöst oder umgangen werden können. Unser Anstrengungen zielen darauf ab, kontinuierlich nach den besten subobtimalsten Zustand zustreben.\\

\subsection{TSU}

Temporär Spielumgebung schaffen

Quelle: ICE Fahrt nach Weihnachten

NFM kann sich sicher selbst beschäftigen, jedoch benötigt es unsere Unterstützung Temporäre Spielumgebungen zu schaffen. Im weitern Verlauf (Vielleicht Zeitabschnitte benennen) kann und wird von uns dazu hinentwickelt, dass es sich eigene TSUs bauen kann. Die Ressourcen Bindung wird sein, diese dem NFM zu Verfügung zu stellen und diese über den gegeben Zeitraum aufrecht zuhalten. Je nach Konzeption, kann ein TSU physischer oder auch mentaler Natur sein. Z.B.: Eine Spielecke, im Raum tanzen, Laut singen. 

Hierbei ist unsere Aufgabe die Interationsfluss mit der Gesellschaft zu managen, um eine TSU für das NFM und die angrenzende Gesellschaft aufrechtzuerhalten und jeweilige Präferenzen gegegenüber NFM und der Gesellschaft zu optimieren.

Z.B.: Freie Entwicklung zu singen, um NFM die Möglichkeit im Zug zu geben, sich auszuprobieren und den Geräusch Pegel so anzupassen, dass andere angrenzen Personen, sich so wohl damit fühlen, sodass diese nicht versuchen den TSU aufzulösen.

\subsection{Mantra}
Die Mantras sind nur Phrasen, welche eine Verlinkung im Kopf herstellen, zu einem größeren Thema oder eine emotionale Hürde im Moment überwinden.


\paragraph{Beste für Dich}
Im Herzen von jedem in der Familie sollte, das der Wunsch vorhanden sein, dass \textit{Beste} für die anderen \gls{p_FM} zu wollen.\\

Dies ist schon in einer zweiter Dynamik schwer, in der mind. dreier Dynamik mit dem \nameref{NFM_O_1} ist dies eine andere Herausforderung. Sich daran zu erinnern, ist jedoch das \textit{Mantra}, an welches sich permanent erinnert werden muss.

\paragraph{Batterien aufladen} Für jeden in der Familie muss der Freiraum gegeben sein, seine Batterien wieder aufzuladen.

\paragraph{Sammlung} 
Nicht alle Mantras sind ausformuliert, wie das obrige. Diese werden hier als Liste aufgeführt. Das Ziel davon ist, diese im Alltag auszuprobieren oder weiter zu vertesten. 

\begin{itemize}
	\item Du bist eine Bereicherung
	\item Eigenständig $\&$ Kompetent
	\item With all that is given to you, you make it your own. (Marvel, Shing Chan)
	\item Gemeinsame Zeit gibt uns Kraft (Z)
\end{itemize}

\subsubsection{Namesgebung}

\begin{itemize}
	\item[Neugeborenes] $[0, 28 \text{Lebenstag}]$
	\item[Saugling] $[0,365 \text{Lebenstag}]$
	\item[Kleinkind] $[1 \text{Lebensjahr}, 3 \text{Lebensjahr}]$
	\item[Kind] $[4 \text{Lebensjahr}, 12 \text{Lebensjahr}]$
	\item[Jugentlicher] $[13 \text{Lebensjahr}, 17 \text{Lebensjahr}]$
	\item[Erwachsener] $[18 \text{Lebensjahr}, \dots )$
\end{itemize}
