\section{Bewertung Situation}

Am besten ist es, sich auszuschreiben. Besonders, wenn nicht klar ist, wo man starten will oder muss. 

In der Situation S_0 wird durch die Person j eine Präferenz Bewertung durchgeführt. Diese findet start im relativen Vergleich. Vielleichtt sollte dies ein Case sein. Der Fakt, dass eine Situation bewertet wird in relation zur Situation S1. 

Wie auch immer, die Situation S_0 wird verglichen zur zukünftigen Situation S1. Dabei ist es egal, ob dies Situation wirklich existiert oder ob diese sich nur vorstellen lässt. *Bei der Bewertung können sich so viele Fehler einschleichen
• Schwer messbare Kenngrößen für einen selber und absolut
• Nicht erreichbare Zwischenschritt
• Größere Hürde der Überwindung.

Diese Situation kann nicht mit dem System ASOS bewertet werden. Es gibt kein Gefühl für die Zukunkft. *Ein Nachenken über die Zukunft kann eine Emonationale Reaktion hervorrufen. Dies ist aber damit verbunden, dass … Interessanter Punkt (Hier brauche ich Hilfe)

Ein Beispiel war, dass die Zukunft nur vom NZOS bewertet werden kann. Zum Beispiel: Hunger kann sich nicht vorgestellt werden, was es bedeutet, dass wir Hunger aufkummulieren über mehrer Tag. Wir können eine Test-Emmotion abrufen, was bedeutet es.
(Hilfe) Kann man eine andere Emmotion geben, welche Hunger zwar simulieren, aber ihn nie wirklich fühlen lassen kann.
Was es es dann Aufsicht, mit dem Gefühl der Zukunftsangst? Wird hier die Bewertung der Zukunft vorgenommen oder ist dies eine separate Bewertung. Gerade hatte ich den Gedanken. 
Die Summultion kann ein Gefühl ausüben. Genau. !!! Es wird sich nicht wirklich vorgestellt, wie die Situation auf einen wirkt, sondern es wird nur empfunden, wie die Simultion empfunden wird. 

so zum Beispiel hat Zukunftsangst damit zutun, wie der Gedanke über die Zukunft ebenfalls einer Bewertung unterläuft. Fuck me. Das bedeutet, dass die Bewertung der Bewertung berücksichtig wird.. Es zählt somit immer nur, dass hier und jetzt. Zum anderen wird bewertet, wie die Bewertung der Zukunft sich anführt. . Ebenso wird bewertet, was die Zukunft für „Situatiosgefühle mit sich bringen kann. Es werden Abschätzungen getroffen, welche Situationen auf einen auf dem Weg dahin erwarten werden.

Wie kann ich erweiterten, 

Die Bewertung von S_1 führt f_A (f_N(S_1)). Diese muss ebenso berücksichtig werden. In S_0 wird f_A(f_N(S_1)) mit in den Entscheidungsprozess mit eingezogen. 

FSDFS e-fax f_A(S_1) ist nicht möglich 

\subsection{Test}
Es geht erstmal darum, zu beschreiben, was getan werden muss.
Vielleicht gibt es andere möglichkeiten, die Situation zu beschrieben.
Aktuell hängt mir noch nach, dass ich gern DPA fertig bekommen möchte, um neue Themen zu beginnen.
Egal, jetzt wird erstmal das Thema beschrieben: Zumindestens die Idee ausgearbeitet.

Zu erst: Gibt es zwei Situation oder gibt es zwei S

\paragraph{Zwei Siutationen} Eigentlich ist es wie bei DPA. Es wird jetzt komplexer und komplexer, obwohl dies überhaupt nicht der Wunsch ist.

Sei $S_t$ eine zu betrachtende Situation.
\footnote{
	Was eine genaue Situation ist, muss noch genauer aus Definiert werden.
}
\footnote{
	Für die erste Besprechung mit Jannik, ist der Fokus darauf, nicht alles auszudefiniere und Lemma zu beweisen, sondern die Grundidee aufzustellen.
}
Für $t=0$ gilt, $S_0$ ist eine Entscheidung oder Umgebung Situation, in welcher sich die Person $P_i$ befindet. 
Mit $t=0$ ist die Gegenwart gemeint. 
Für $t=1$ gilt, $S_1$ ist eine theoretische Entscheidung oder Umgebungssituation in der Zukunft.
\begin{itemize}
	\item Es gibt keinen Bezug oder Aktionen die getroffen werden können, welche eine direkt Auswirkung auf die nahe Gegenwart haben.
	\item Es gibt also eine Bereich, um die Gegenwart, welche eine Trennung verursacht. Die Frage ist, eine Trennung von was?
	\item Hier geht es darum, dass das aktuelle Gefühl nicht durch die Entscheidung geändert werden kann.
	\item Eine Entscheidungslokalität kann hier zum Einsatz kommen.
\end{itemize}