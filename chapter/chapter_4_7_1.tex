
\section{Meaning of a Rubric}
\label{sec:Rubric}

\paragraph{Coherent Decision Framework}

A rubric for a given topic is a collection of principals, goals or a thesis under a \gls{p_CDF}.\\

The process of integrating items of a rubric under \gls{p_CDF} also defines the topic itself. Since there is no natural boundary where a topic can end, defining these boundaries is part of the process of developing a rubric.\\

In this context, \textit{coherent} means that the parts of the rubric are connected with each other. The process involves testing parts of the rubric against each other to ensure they are:
\begin{description}
    \item[Aligned] supporting the main topic's goals or thesis.
    \item[Non-Contradictory] not in contradiction with other parts; if contradictions exist, the rubric should address how to resolve them.
    \item[Integrated] ideally, many parts of the decision framework should be integrated with each other, thereby reinforcing alignment towards the goal or thesis.
\end{description}

\paragraph{Communicated clearly and consistently:}
To handle the complexity of the created structure, a quality criterion is that a \textit{rubric} must be communicated clearly and consistently.

\paragraph{Relationship between a Strategy}

Loosely speaking, a strategy is a rubric plus specific actions (Maßnahmen).

The \textit{main two differences} are:
\begin{itemize}
	\item A strategy contains \textit{specific actions/ decision (Maßnahmen/ Entscheidungen)} that are already decided to be taken.
	\item It is more strict in defining the boundaries of a topic. The analysis of the \textit{playing field} assesses the topic or playing field against other actors.
\end{itemize} 

A strategy outlines specific actions based on set goals and analyzes them and the game field where they should play out. A rubric provides a coherent framework to make choices along the line. Along those lines, a strategy can contain the same elements as a rubric. These are then integrated into a \gls{p_CDF} for continuity.
