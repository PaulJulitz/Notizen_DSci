\section{Data Preprocessing: Categorial Values}
%Quelle: Working with Predictive Analytics} 
In diesem Themen-Block wurden die Kategorischen Varialben aufgegriffen, und die Option des One Hot und Label Encoders.

Prozesse und Algorithmen im \gls{ML} Bereich benötigen als Input numerische Werte. Daten diese dies nicht bieten können, müssen in solche kodiert werden.

Es gibt dafür zwei bestehende Methoden im sklearn Paket:
\begin{itemize}
	\item 
	\begin{lstlisting}[style=python]
		from sklearn import preprocessing
		le = preprocessing.LabelEncoder()	 # Create a class object
	\end{lstlisting}
	\item 
	\begin{lstlisting}[style=python]
		
	\end{lstlisting}
\end{itemize}

\paragraph{Label Encoder}
Der Preprocessor von \gls{g_scikit} bietet eine Classe an, welche Kategorische Werte eines Arrays in numerische umwandelt.
Dies ist notwendig, weil ML-Modelle nur numerische Werte einlesen.
Haben die Werte eine ordinale Struktur, kann ein ML Modell diese abbilden. Ist dies jedoch nicht vorhanden, so bietet One Hot Encoder eine bessere Struktur an.

\begin{lstlisting}[style=python]
	#################### Python Libraries
	import pandas as pd
	import numpy as np
	from sklearn.model_selection import train_test_split
	from sklearn.impute import SimpleImputer
	import matplotlib.pyplot as plt
	
	# Label Encoder
	from sklearn.preprocessing import OneHotEncoder
	from sklearn.preprocessing import LabelEncoder
	from sklearn.compose import ColumnTransformer
	
	#################### Import Date
	data = pd.read_csv("insurance.csv")
	
	#################### EDA
	
	#print(data.head(3))
	#print(data.describe())
	#print(data["bmi"].unique())
	#data_nan = data.isnull().sum()
	#print(data_nan)
	
	#################### Data Prepartion
	
	data["bmi"].fillna(0, inplace = True) # replace None
	#print(data.head(3))
	data_mod = data
	
	#################### Label Encoding
	LE = LabelEncoder()
	data_mod["sex"] = LE.fit_transform(data["sex"]) # input is an array
	data_mod["region"] = LE.fit_transform(data["region"])
	data_mod["smoker"] = LE.fit_transform(data["smoker"])
	print(data_mod.iloc[:,1:].head())	
\end{lstlisting}

Im Klassenobjekt werden die Kategorien wie weitere Funktion zum Input Array gespeichert. Wir das Klassenobjekt auf eine anderes Array angesetzt, so werden die vorherigen Daten überschrieben.
\begin{lstlisting}[style=python]
	LE = LabelEncoder()
	data_mod["sex"] = LE.fit_transform(data["sex"]) # input is an array
	data_mod["region"] = LE.fit_transform(data["region"])
	data_mod["smoker"] = LE.fit_transform(data["smoker"])
	print(data_mod.iloc[:,1:].head())
	
	# Inverse encoder
	print(LE.classes_)
	#data_mod["New"] = LE.inverse_transform(data_mod["region"] #Fuehrt zum Fehler
	data_mod["New"] = LE.inverse_transform(data_mod["smoker"]
\end{lstlisting}

\paragraph{One Hot Encoder}
Um eine Ordnung in den transformierten Daten zu vermeiden, wenn keine existiert, werden Dummy Variablen angelegt. \\

Im ersten Schritt wird der Konstruktor geschaffen, dann die jeweiligen Operationen durchgeführt.
\begin{lstlisting}[style=python]
	O_OHE = OneHotEncoder(categorial_feature=[0])
	O_OHE.fit_Transform1	
\end{lstlisting}
Das label-encodedet Array wird aufsplittet in die Anzahl Spalten wie das Array eindeutige Kategorien hat.
