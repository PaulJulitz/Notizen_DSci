%%% Seitenkonfigurationen und Formatierung
	\usepackage{fullpage}
	\usepackage[ngerman]{babel}	
	\usepackage[utf8]{inputenc}	
	\usepackage[top=1.5cm, left=3cm, right=3cm,bottom=2.5cm,headsep=1.5cm]{geometry}  
	\usepackage{microtype} % verbesserter Randausgleich


%%%
\usepackage{scrlayer-scrpage} % Verwendung der Fuß- und Kopfzeile
		\pagestyle{scrheadings}
		%\clearscrheadfoot
		%\chead[{\includegraphics[width=1\linewidth]{attachment/x_titlepage/balken_dsci}}]{\includegraphics[width=1\linewidth]{attachment/x_titlepage/balken_dsci}}
		\ofoot[\pagemark]{\pagemark}
	\renewcommand{\thesection}{\arabic{section}} % Entfernt die Null vor der Section Nummerierung
	\setcounter{tocdepth}{3} % Ändert die Verästelung, wie weit ToC gehen soll.
	\setcounter{secnumdepth}{3} % Ändert die Nummerirung für die verschiedenen Ebenen.
	\setlength\parindent{2pt} % Noindent im gesamten Dokument

%%% Farben und Fonts
		\usepackage[dvipsnames]{xcolor}
			% Style 
			\definecolor{mainone}{RGB}{5,56,110} 
			\definecolor{maintwo}{RGB}{0,63,125} 
			\definecolor{mainthree}{RGB}{7,82,161}
			\definecolor{mainfour}{RGB}{0,115,228}
			% Listings Style
			\definecolor{vbagreen}{RGB}{0,128,0}
			\definecolor{vbablue}{RGB}{0,0,128}
			% More colors
			\definecolor{mysticlight}{RGB}{222,228,240}
		\usepackage{sectsty} % Einfärbung der Section-Titel
			\partfont{\color{mainone}}  
			\sectionfont{\color{mainone}}
			\sectionfont{\color{mainthree}}
			\subsectionfont{\color{mainthree}}

%%% Symbole und Grafiken
	\usepackage{amsmath,amsfonts, amssymb,bm}
		\numberwithin{equation}{section} % Indezierung der Align-Umgebung
	\usepackage{graphicx} % für den Befehl 	includegraphics
	\usepackage{mathtools} % Verbesserung von amsmath; Nutzung von Math-Command außerhalb der Math-Env
	\usepackage{eurosym} % Eurozeichen
	\usepackage{siunitx} % SI-Einheiten
		\sisetup{
			per-mode=fraction, 
			group-digits=true,
			group-separator={\,},
			%decimalsymbol=comma,
			fraction-function=\tfrac,
			binary-units=true,
			number-math-rm = \ensuremath 
		}
	\usepackage{float} 	%Erzwingen der figure [H]
	\usepackage{pgf-umlcd} %Class Diagramm, UML Diagramm
		\renewcommand{\umlfillcolor}{white}
		\renewcommand{\umldrawcolor}{black}
		\renewcommand{\umltextcolor}{black}

%%% Appendix
	\usepackage[pdfencoding=auto,
			%hidelinks,
			colorlinks = true,
			linkcolor = mainone, % Farbe für Interneverweise
			urlcolor = mainone, % Farbe für URL Link
			citecolor= mainone % Farbe für Citationen 
			]{hyperref} % Hyperlinks; hidelinks - vorherige Option
	\usepackage{appendix} % Appendix
		% Option: page - Front page
		% Option: toc - Eintrag in ToC
		% Option: titletoc - 
%		\renewcommand{\appendixtocname}{} % Ändert Chapter-Namen in toc
%		\renewcommand{\appendixname}{Appendizes} % Ändert Chapter-Namen im Dokument - KLAPPT NICHT
	\usepackage{acronym} % Abkürzungsverzeichnis
	\usepackage[automake, % automake erstellt automatisch eine .gls datei
				acronym, % Acronym Packet; Unterschied zu Acronym ist, dass glossaries eine Sortierungsfunktion hat
				style=indexgroup
			]{glossaries} % Glossar Package
		\renewcommand{\glossarysection}[2][]{} % Unterdrückt den Titel des Glossars

%%% Sonstiges
	\usepackage{comment} % Auskommentieren
	\usepackage{pdfpages} % Einzelne Seiteneinbindung der PDFs
		% Option: nup=Spalten x Zeilen. Rasterdarstellung