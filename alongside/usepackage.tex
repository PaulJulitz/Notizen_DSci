%%% Seitenkonfigurationen und Formatierung
	\usepackage{fullpage}
	\usepackage[ngerman]{babel}	
	%\usepackage[utf8]{inputenc} % Problem with ipythonnb enviroment
	\usepackage[top=1.5cm, left=3cm, right=3cm,bottom=2.5cm,headsep=1.5cm]{geometry}  
	\usepackage{microtype} % verbesserter Randausgleich


%%%
\usepackage{scrlayer-scrpage} % Verwendung der Fuß- und Kopfzeile
		\pagestyle{scrheadings}
		%\clearscrheadfoot
		%\chead[{\includegraphics[width=1\linewidth]{attachment/x_titlepage/balken_dsci}}]{\includegraphics[width=1\linewidth]{attachment/x_titlepage/balken_dsci}}
		\ofoot[\pagemark]{\pagemark}
	\renewcommand{\thesection}{\arabic{section}} % Entfernt die Null vor der Section Nummerierung
	\setcounter{tocdepth}{3} % Ändert die Verästelung, wie weit ToC gehen soll.
	\setcounter{secnumdepth}{3} % Ändert die Nummerirung für die verschiedenen Ebenen.
	\setlength\parindent{2pt} % Noindent im gesamten Dokument

%%% Farben und Fonts
%		\usepackage{fontspec} % Lulatex Font:Calibri
%			\setmainfont{Calibri}
% Set colors
\usepackage[dvipsnames]{xcolor}
	% Style 
	\definecolor{mainone}{RGB}{5,56,110} 
	\definecolor{maintwo}{RGB}{0,63,125} 
	\definecolor{mainthree}{RGB}{0,63,125} 
	\definecolor{mainfour}{RGB}{0,115,228}
	% Listings Style
	\definecolor{vbagreen}{RGB}{0,128,0}
	\definecolor{vbablue}{RGB}{0,0,128}
	\definecolor{mysticlight}{RGB}{222,228,240} % Background color lislisting
	\definecolor{light-gray}{gray}{0.95} % background color jupter notebook
% Einfärbung der Section-Titel			
\usepackage{sectsty} 
	\partfont{\color{mainone}}  
	\sectionfont{\color{mainone}}
	\subsectionfont{\color{mainthree}}
	\subsubsectionfont{\color{mainthree}}

%%%% Make coloring environments %%%%%%%%%%%%%%%%%%%%
\usepackage[many]{tcolorbox}

%%% Creating theorems environments %%
\tcbuselibrary{theorems}

\makeatletter
	% add tcblistings to \jobname.lol (list of listings)
\tcbset{
	addtolol/.style={list entry={\kvtcb@title},
	add to list={lol}{subsection}},
}
\makeatother

\newtcbtheorem[% <init option>
	auto counter,
	number within=part,
	list inside=ListDefinition
	]{definition}%(name)
	{Definition}%(display name)
	{colback=gray!5,colframe=mainthree!90,fonttitle=\bfseries}%(option)
	{Def}%(prefix) for refernce

\newtcbtheorem[% <init option>
	auto counter,
	number within=part,
	list inside=ListLemmaDefinition
	]{lemma-definition}%(name)
	{Definition}%(display name)
	{colback=gray!5,colframe=gray!90,fonttitle=\bfseries,}%(option)
	{Def}%(prefix) for reference

%%% Jupter Notebook entries %%
				%% Environment for iPython Notebook
				% Reference: https://tex.stackexchange.com/questions/223465/ipython-notebook-cells-with-listings
				% Using each IPython cell with: \begin{ipythonnb} ... \end{ipythonnb}
				% Linked in usepackage.tex
		% Many is used for lstings
\tcbuselibrary{listings} % this is not the usepackage \usepackage{listings}
% the space reserved between for the ``In'' numbers and the code
\newlength\inwd
\setlength\inwd{1.3cm}

\newcounter{ipythcntr}

\newtcblisting{ipythonnb}[1][\theipythcntr]{
  enlarge left by=\inwd,
  width=\linewidth-\inwd,
  enhanced,
  boxrule=0.4pt,
  colback=light-gray,
  listing only,
  top=0pt,
  bottom=0pt,
  overlay={
    \node[
      anchor=north east,
      text width=\inwd,
      font=\footnotesize\ttfamily\color{blue!50!black},
      inner ysep=2mm,
      inner xsep=0pt,
      outer sep=0pt
      ] 
      at (frame.north west)
      {\stepcounter{ipythcntr}In [#1]:};
  }
  listing options={
    basicstyle=\footnotesize\ttfamily,
    language=python,
    escapechar=¢,
    showstringspaces=false,
  },
}




%%% Symbole und Grafiken
	\usepackage{amsmath,amsfonts, amssymb,bm}
		\numberwithin{equation}{section} % Indezierung der Align-Umgebung
	\usepackage{graphicx} % für den Befehl 	includegraphics
	\usepackage{mathtools} % Verbesserung von amsmath; Nutzung von Math-Command außerhalb der Math-Env
	\usepackage{eurosym} % Eurozeichen
	\usepackage{siunitx} % SI-Einheiten
		\sisetup{
			per-mode=fraction, 
			group-digits=true,
			group-separator={\,},
			%decimalsymbol=comma,
			fraction-function=\tfrac,
			binary-units=true,
			number-math-rm = \ensuremath 
		}
	\usepackage{float} 	%Erzwingen der figure [H]
	\usepackage{pgf-umlcd} %Class Diagramm, UML Diagramm
		\renewcommand{\umlfillcolor}{white}
		\renewcommand{\umldrawcolor}{black}
		\renewcommand{\umltextcolor}{black}


	
	\usepackage{tikz} % Grafikgestaltung
		\usetikzlibrary{calc}
		\usetikzlibrary{arrows, arrows.meta}
		\usetikzlibrary{fit}% <-- new
		\tikzset{%
			level 1/.style={level distance=3.5cm, sibling distance=4.5cm},
			level 2/.style={level distance=3.5cm, sibling distance=2cm},
			bag/.style={text width=50pt, text centered},
			end/.style={circle, minimum width=3pt, fill, inner sep=0pt},
		}
	
	% XY-Diagramm
	\usepackage[all]{xy}
	
	% Plot generator
	\usepackage{pgfplots}
	%\pgfplotsset{width=7cm,compat=1.9}
	
	% We will externalize the figures
	%\usepgfplotslibrary{external}
	%\tikzexternalize
	%- Entscheidungsbaum; Struktur, Folder Aufbau
	\usepackage[edges]{forest}
	\def\Size{.9pt}
	\tikzset{pics/folder/.style={code={%
				\node[inner sep=-5pt, minimum size=#1](-foldericon){};
				\node[folder style, inner sep=0pt, minimum width=0.6*#1, minimum height=0.6*#1, above right, xshift=-0.25*#1] at (-foldericon.west){};
				\node[folder style, inner sep=4pt, minimum size=#1, xshift=-0.4*#1] at (-foldericon.center){};}
		},
		pics/folder/.default={10pt},
		folder style/.style={draw=gray,top color=mainone,bottom color=mainone}
	}
	
	\forestset{is file/.style={edge path'/.expanded={%
				([xshift=\forestregister{folder indent}]!u.parent anchor) |- (.child anchor)},
			inner sep=1pt},
		this folder size/.style={edge path'/.expanded={%
				([xshift=\forestregister{folder indent}]!u.parent anchor) |- (.child anchor) pic[solid]{folder=#1}}, inner ysep=0.6*#1},
		folder tree indent/.style={before computing xy={l=#1}},
		folder icons/.style={folder, this folder size=#1, folder tree indent=3*#1},
		folder icons/.default={10pt},
	}
	
	
%%% Tabellen
	\usepackage{multirow}
		\renewcommand{\arraystretch}{1.5}	
		% ? footskip=1.2cm
	\usepackage{hhline} % geht anders mit vertikalen Linien um, als hline
	\usepackage{colortbl} % Einfärbung von Zellen 
	\usepackage{booktabs} % \toprule, \midrule und \bottomrule
	

%%% Appendix
	\usepackage[pdfencoding=auto,
			%hidelinks,
			colorlinks = true,
			linkcolor = mainone, % Farbe für Interneverweise
			urlcolor = mainone, % Farbe für URL Link
			citecolor= mainone % Farbe für Citationen 
			]{hyperref} % Hyperlinks; hidelinks - vorherige Option
	\usepackage{appendix} % Appendix
		% Option: page - Front page
		% Option: toc - Eintrag in ToC
		% Option: titletoc - 
%		\renewcommand{\appendixtocname}{} % Ändert Chapter-Namen in toc
%		\renewcommand{\appendixname}{Appendizes} % Ändert Chapter-Namen im Dokument - KLAPPT NICHT

%	\usepackage[printonlyused]{acronym} % Abkürzungsverzeichnis
	% This package is not been needed, because glossaries takes care of it
	% Option: printonlyused to display acronyms which are used.
	\usepackage[automake, % automake erstellt automatisch eine .gls datei
				acronym, % Acronym Packet; Unterschied zu Acronym ist, dass glossaries eine Sortierungsfunktion hat
				style=indexgroup
			]{glossaries} % Glossar Package
		\renewcommand{\glossarysection}[2][]{} % Unterdrückt den Titel des Glossars
		%\glsaddall % Der Befehl fügt alle Glossary-Einträge in das Glossar ein. This has to be place inside the document.

%%% Sonstiges
	\usepackage{comment} % Auskommentieren
	\usepackage{pdfpages} % Einzelne Seiteneinbindung der PDFs
		% Option: nup=Spalten x Zeilen. Rasterdarstellung
		
%% BibLaTeX:

%% in der Präambel:
\usepackage[backend=biber, %% Hilfsprogramm "biber" (statt "biblatex" oder "bibtex")
style=authoryear, %% Zitierstil (siehe Dokumentation)
natbib=true, %% Bereitstellen von natbib-kompatiblen Zitierkommandos
hyperref=true, %% hyperref-Paket verwenden, um Links zu erstellen
]{biblatex}




