\usepackage{listings} % Include the listings-package

			\lstdefinestyle{VBA}
			{
				language={[Visual]Basic},
				commentstyle=\color{vbagreen},  % comment style
				keywordstyle=\color{vbablue},   % keyword style
				basicstyle=\footnotesize\ttfamily, % the size of the fonts that are used for the code
				firstnumber=1,                	% start line enumeration with line 1
				frame=tb,	                	% adds a frame around the code (single, tb,...)	
				numbers=left,                   % where to put the line-numbers; possible values are (none, left, right)
				numbersep=5pt,                  % how far the line-numbers are from the code
				numberstyle=\tiny\color{gray},	% the style that is used for the line-numbers
				rulecolor=\color{black},        % if not set, the frame-color may be changed on line-breaks within not-black text (e.g. comments (green here))
				%keepspaces=false,              % keeps spaces in text, useful for keeping indentation of code (possibly needs columns=flexible)				
  				showspaces=false,               % show spaces everywhere adding particular underscores; it overrides 'showstringspaces'
				showstringspaces=false,         % underline spaces within strings only
				tabsize=2,		                % sets default tabsize to 2 spaces
				columns=fixed,					% Using fixed column width (for e.g. nice alignment)
				breaklines=true,				% sets if atomatic line breaking
				escapeinside={(*@}{@*)},		% if you want to add LaTeX within your code
				morecomment=[l]{//},			% add comment syntax
		}
			\lstdefinestyle{C++}
		{
				language=C++,
				basicstyle=\footnotesize\ttfamily, % the size of the fonts that are used for the code
				keywordstyle=\color{blue}\ttfamily,
				stringstyle=\color{red}\ttfamily,
				commentstyle=\color{green}\ttfamily,
				morecomment=[l][\color{magenta}]{\#}
				firstnumber=1,                	% start line enumeration with line 1
				frame=tb,	                	% adds a frame around the code (single, tb,...)	
				numbers=left,                   % where to put the line-numbers; possible values are (none, left, right)
				numbersep=5pt,                  % how far the line-numbers are from the code
				numberstyle=\tiny\color{gray},	% the style that is used for the line-numbers
				rulecolor=\color{black},  
		}
			\lstdefinestyle{M}
		{
			language=SQL,
			commentstyle=\color{vbagreen},  % comment style
			keywordstyle=\color{vbablue},   % keyword style
			basicstyle=\footnotesize\ttfamily, % the size of the fonts that are used for the code
			firstnumber=1,                	% start line enumeration with line 1
			frame=tb,	                	% adds a frame around the code (single, tb,...)	
			numbers=left,                   % where to put the line-numbers; possible values are (none, left, right)
			numbersep=5pt,                  % how far the line-numbers are from the code
			numberstyle=\tiny\color{gray},	% the style that is used for the line-numbers
			rulecolor=\color{black},        % if not set, the frame-color may be changed on line-breaks within not-black text (e.g. comments (green here))
			%keepspaces=false,              % keeps spaces in text, useful for keeping indentation of code (possibly needs columns=flexible)				
			showspaces=false,               % show spaces everywhere adding particular underscores; it overrides 'showstringspaces'
			showstringspaces=false,         % underline spaces within strings only
			tabsize=2,		                % sets default tabsize to 2 spaces
			columns=fixed,					% Using fixed column width (for e.g. nice alignment)
			breaklines=true,				% sets if atomatic line breaking
			escapeinside={(*@}{@*)},		% if you want to add LaTeX within your code
			morekeywords={let},				% add keywords
			morecomment=[l]{//}				% add comment syntax
		}
			\lstdefinestyle{DAX}
		{
			language={[Visual]Basic},
			commentstyle=\color{vbagreen},  % comment style
			keywordstyle=\color{vbablue},   % keyword style
			basicstyle=\footnotesize\ttfamily, % the size of the fonts that are used for the code
			firstnumber=1,                	% start line enumeration with line 1
			frame=tb,	                	% adds a frame around the code (single, tb,...)
			numbers=left,                   % where to put the line-numbers; possible values are (none, left, right)
			numbersep=5pt,                  % how far the line-numbers are from the code
			numberstyle=\tiny\color{gray},	% the style that is used for the line-numbers
			rulecolor=\color{black},        % if not set, the frame-color may be changed on line-breaks within not-black text (e.g. comments (green here))
			%keepspaces=false,              % keeps spaces in text, useful for keeping indentation of code (possibly needs columns=flexible)				
			showspaces=false,               % show spaces everywhere adding particular underscores; it overrides 'showstringspaces'
			showstringspaces=false,         % underline spaces within strings only
			tabsize=2,		                % sets default tabsize to 2 spaces
			columns=fixed,					% Using fixed column width (for e.g. nice alignment)
			breaklines=true,				% sets if atomatic line breaking
			escapeinside={(*@}{@*)},			% if you want to add LaTeX within your code
			morekeywords={Calculate,All,date,Filter,Average,Rank.EQ, RankX,Count,CountX,SumX}
		}
	
			\lstdefinestyle{SQL}
		{
			language=SQL,
			commentstyle=\color{vbagreen},  % comment style
			keywordstyle=\color{vbablue},   % keyword style
			basicstyle=\footnotesize\ttfamily, % the size of the fonts that are used for the code
			firstnumber=1,                	% start line enumeration with line 1
			frame=tb,	                	% adds a frame around the code (single, tb,...)	
			numbers=left,                   % where to put the line-numbers; possible values are (none, left, right)
			numbersep=5pt,                  % how far the line-numbers are from the code
			numberstyle=\tiny\color{gray},	% the style that is used for the line-numbers
			rulecolor=\color{black},        % if not set, the frame-color may be changed on line-breaks within not-black			
			showspaces=false,               % show spaces everywhere adding particular underscores; it overrides 'showstringspaces'
			showstringspaces=false,         % underline spaces within strings only
			tabsize=2,		                % sets default tabsize to 2 spaces
			columns=fixed,					% Using fixed column width (for e.g. nice alignment)
			breaklines=true,				% sets if atomatic line breaking
			morecomment=[l]{//}	,			% add comment syntax
			morekeywords={OFFSET}
		}
			\lstdefinestyle{Qlik}
		{
				language=SQL,
				commentstyle=\color{vbagreen},  % comment style
				keywordstyle=\color{vbablue},   % keyword style
				basicstyle=\footnotesize\ttfamily, % the size of the fonts that are used for the code
				firstnumber=1,                	% start line enumeration with line 1
				frame=tb,	                	% adds a frame around the code (single, tb,...)	
				numbers=left,                   % where to put the line-numbers; possible values are (none, left, right)
				numbersep=5pt,                  % how far the line-numbers are from the code
				numberstyle=\tiny\color{gray},	% the style that is used for the line-numbers
				rulecolor=\color{black},        % if not set, the frame-color may be changed on line-breaks within not-black text (e.g. comments (green here))
				%keepspaces=false,              % keeps spaces in text, useful for keeping indentation of code (possibly needs columns=flexible)				
				showspaces=false,               % show spaces everywhere adding particular underscores; it overrides 'showstringspaces'
				showstringspaces=false,         % underline spaces within strings only
				tabsize=2,		                % sets default tabsize to 2 spaces
				columns=fixed,					% Using fixed column width (for e.g. nice alignment)
				breaklines=true,				% sets if atomatic line breaking
				escapeinside={(*@}{@*)},		% if you want to add LaTeX within your code
				morekeywords={LOAD},				% add keywords
				morecomment=[l]{//}				% add comment syntax
		}
	
			\lstdefinestyle{git}
		{
				language=bash,
				basicstyle=\footnotesize\ttfamily, % the size of the fonts that are used for the code
				backgroundcolor = \color{mysticlight}, % Background color
				keywordstyle=\color{vbablue},   % keyword style
				commentstyle=\color{vbagreen},  % comment style
				firstnumber=1,                	% start line enumeration with line 1
				frame=tb,	                	% adds a frame around the code (single, tb,...)	
				numbers=left,                   % where to put the line-numbers; possible values are (none, left, right)
				numbersep=5pt,                  % how far the line-numbers are from the code
				numberstyle=\tiny\color{gray},	% the style that is used for the line-numbers
				rulecolor=\color{black},        % if not set, the frame-color may be changed on line-breaks within not-black			
				showspaces=false,               % show spaces everywhere adding particular underscores; it overrides 'showstringspaces'
				showstringspaces=false,         % underline spaces within strings only
				tabsize=2,		                % sets default tabsize to 2 spaces
				columns=fixed,					% Using fixed column width (for e.g. nice alignment)
				breaklines=true,				% sets if atomatic line breaking
				morecomment=[l]{///}				% add comment syntax
		}
			\lstdefinestyle{CMD}
		{
				language=bash,
				basicstyle=\color{white}\footnotesize\ttfamily, % the size of the fonts that are used for the code
				xleftmargin=.05\textwidth,
				xrightmargin=.05\textwidth,
				escapeinside={<@}{@>},
				backgroundcolor = \color{black}, % Background color
				keywordstyle=\color{yellow},   % keyword style
				commentstyle=\color{vbagreen},  % comment style
				firstnumber=1,                	% start line enumeration with line 1
				frame=tb,	                	% adds a frame around the code (single, tb,...)	
				numbers=left,                   % where to put the line-numbers; possible values are (none, left, right)
				numbersep=5pt,                  % how far the line-numbers are from the code
				numberstyle=\tiny\color{gray},	% the style that is used for the line-numbers
				rulecolor=\color{black},        % if not set, the frame-color may be changed on line-breaks within not-black			
				showspaces=false,               % show spaces everywhere adding particular underscores; it overrides 'showstringspaces'
				showstringspaces=false,         % underline spaces within strings only
				tabsize=2,		                % sets default tabsize to 2 spaces
				columns=fixed,					% Using fixed column width (for e.g. nice alignment)
				breaklines=true,				% sets if atomatic line breaking
				morekeywords={conda, +, flake8, mypy, pip, python},		% add keywords
				morecomment=[l]{///}			% add comment syntax
		}
		\lstdefinestyle{Config}
		{
			basicstyle=\ttfamily\small,
			columns=fullflexible,
			morecomment=[s][\color{Orchid}\bfseries]{[}{]},
			morecomment=[l]{\#},
			morecomment=[l]{;},
			firstnumber=1,                	% start line enumeration with line 1
			frame=single,	                	% adds a frame around the code (single, tb,...)	
			numbers=left,                   % where to put the line-numbers; possible values are (none, left, right)
			numbersep=5pt,                  % how far the line-numbers are from the code
			numberstyle=\tiny\color{gray},	% the style that is used for the line-numbers
			%rulecolor=\color{black},        % if not set, the frame-color may be changed on line-breaks within not-black	
			commentstyle=\color{gray}\ttfamily,
			morekeywords={},
			otherkeywords={:},
			xleftmargin=3cm,
			xrightmargin=3cm,
			keywordstyle={\color{blue}\bfseries}
		}
% Default fixed font does not support bold face
\DeclareFixedFont{\ttm}{T1}{txtt}{m}{n}{9}  % for normal
		\lstdefinestyle{python}
	{
		language=python,
		basicstyle=\ttm, % the size of the fonts that are used for the code
		xleftmargin=.05\textwidth,
		xrightmargin=.05\textwidth,
		framextopmargin=3pt,
		framexbottommargin=4pt, 
		escapeinside={<@}{@>},
		backgroundcolor = \color{light-gray}, % Background color
		keywordstyle=\color{vbablue},   % keyword style
		commentstyle=\color{gray}\ttfamily,  % comment style
		firstnumber=1,                	% start line enumeration with line 1
		framesep=0.5em,
		%frameround=tttt,
		%frame=single,	                	% adds a frame around the code (single, tb,...)	
		numbers=left,                   % where to put the line-numbers; possible values are (none, left, right)
		numbersep=5pt,                  % how far the line-numbers are from the code
		numberstyle=\tiny\color{gray},	% the style that is used for the line-numbers
		%rulecolor=\color{black},        % if not set, the frame-color may be changed on line-breaks within not-black			
		showspaces=false,               % show spaces everywhere adding particular underscores; it overrides 'showstringspaces'
		showstringspaces=false,         % underline spaces within strings only
		tabsize=2,		                % sets default tabsize to 2 spaces
		columns=fixed,					% Using fixed column width (for e.g. nice alignment)
		breaklines=true,				% sets if atomatic line breaking
		morekeywords={}%,				% add keywords
	%	morecomment=[l]{///}			% add comment syntax
	}